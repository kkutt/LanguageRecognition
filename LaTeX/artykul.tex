\documentclass[journal]{IEEEtran}

%\usepackage{polski}
\usepackage[T1]{fontenc}
\usepackage[utf8]{inputenc}
\usepackage[pdftex]{graphicx}
\graphicspath{/img/}

% correct bad hyphenation here
\hyphenation{op-tical net-works semi-conduc-tor}


\begin{document}
%
% paper title
% can use linebreaks \\ within to get better formatting as desired
\title{Rozpoznawanie języka tekstu \\ na podstawie częstotliwości liter}

\author{Krzysztof~Kutt, Michał~Nowak
\thanks{K. Kutt i M. Nowak, Katedra Automatyki, Wydział Elektrotechniki, Automatyki, Informatyki i Elektroniki,
Akademia Górniczo-Hutnicza, Kraków, Polska, e-mail: kkutt@student.agh.edu.pl, rial@student.agh.edu.pl}}

\markboth{Sztuczne Sieci Neuronowe, semestr letni 2011/2012, prowadzący: mgr inż. Tomasz Orzechowski}%
{}

% make the title area
\maketitle


\begin{abstract}
TODO: Tutaj leci abstrakt
\end{abstract}

% Note that keywords are not normally used for peerreview papers.
\begin{IEEEkeywords}
sieć neuronowa, język, rozpoznawanie.
\end{IEEEkeywords}


\section{Wprowadzenie}
% The very first letter is a 2 line initial drop letter followed
% by the rest of the first word in caps.
% 
% form to use if the first word consists of a single letter:
% \IEEEPARstart{A}{demo} file is ....
% 
% form to use if you need the single drop letter followed by
% normal text (unknown if ever used by IEEE):
% \IEEEPARstart{A}{}demo file is ....
% 
% Some journals put the first two words in caps:
% \IEEEPARstart{T}{his demo} file is ....
% 
% Here we have the typical use of a "T" for an initial drop letter
% and "HIS" in caps to complete the first word.
%
% You must have at least 2 lines in the paragraph with the drop letter
% (should never be an issue)

\IEEEPARstart{S}{ztuczne} sieci neuronowe to bardzo uproszczone modele ludzkiego mózgu.
Każda z sieci składa się z setek, albo tysięcy sztucznych neuronów, stworzonych na wzór naturalnego
neuronu. Sztuczny neuron zbiera sygnały wejścia (poprzez 'dendryty'), dokonuje na nich odpowiedniej
transformacji i zwraca odpowiednią wartość na wyjściu ('akson'). W zależności od budowy sieci (układu
neuronów) oraz od wykorzystywanej przez neurony funkcji aktywacji, wyróżniane są różne rodzaje sieci~\cite{tad:sieci}.

\subsection{Rodzaje sieci}
Najważniejsze typy sieci neuronowych to sieci jednokierunkowe, sieci rekurencyjne i~sieci Kohonena~\cite{wiki:sieci}.

\subsubsection{Sieci jednokierunkowe}
W tych sieciach neurony są ułożone warstwowo. Sygnały przechodzą od wejścia do wyjścia sieci, przez wszystkie jej
warstwy, bez nawrotów (bez sprzężenia zwrotnego). Rozwiązują dosyć szeroką klasę problemów.

\subsubsection{Sieci rekurencyjne}
Sieci ze sprzężeniem zwrotnym. Połączenia między neuronami stanowią graf z cyklami. Stosowane są np do rozwiązywania
problemów minimalizacji, przykładowo: problemu komiwojażera.

\subsubsection{Sieci Kohonena}
Sieć, która dopasowuje swoją strukturę przestrzenną do zbioru danych (mapa). Uczona bez nauczyciela.


\subsection{Zastosowania sieci}
Sztuczne sieci neuronowe, ze względu na fakt podawania jedynie przybliżonych wyników, nie nadają się do obliczania
dokładnych wartości. Za to, dzięki 'umiejętności' uczenia się, nadają się do rozwiązywania problemów słabo określonych,
do których nie jesteśmy w stanie stworzyć algorytmów. Sieć w trakcie procesu uczenia się może wykryć zależności,
których nie jesteśmy w stanie dostrzec.

Przykładowymi problemami, do których sieci neuronowe są odpowiednimi
narzędziami, są: rozpoznawanie pisma, przetwarzanie obrazu (np w poszukiwaniu ukrtych jednostek wojskowych, czy
podejrzanych pakunków na lotniskach), prognozowanie cen.


\subsection{Rozpoznawanie języka zadanego tekstu}
W niniejszym artykule przedstawiony zostaje inny, ciekawy problem: rozpoznawanie języka zadanego tekstu.

Wg projektu ``Ethnologue'', na świecie istnieje prawie 7.5 tysiąca języków~\cite{ethnologue}. Narzędzie,
które byłoby w stanie rozpoznać z jakim językiem mamy doczynienia, byłoby bardzo użyteczne - dzięki niemu,
można np ustalić jakiego tłumacza potrzebujemy. Wizją przyszłości jest stworzenie automatycznego tłumacza,
który najpierw rozpoznaje mowę naszego rozmówcy, następnie rozpoznaje język, tłumaczy go, a na samym końcu
syntezuje mowę, byśmy mogli w naszym ojczystym języku usłyszeć słowa rozmówcy.

Do zadania tego można podejść na wiele sposobów~\cite{smykowski:jak_google_rozpoznaje}.

\subsubsection{Alfabet}
Pierwszą przesłanką może być wykorzystywany alfabet. Wiele języków posiada charakterystyczne dla siebie znaki
diakrytyczne, jak np język polski i 'ogonki'.

\subsubsection{Charakterystyczne wyrazy}
W przypadku takich samych alfabetów, decyzję o języku można podjąć, zwracając uwagę na charakterystyczne dla
danego języka wyrazy, czy końcówki. Duże zestawienie takich prostych reguł znajduje się w serwisie Wikipedia~\cite{wiki:langs}.

\subsubsection{Częstotliwość liter}
Można również ustalić częstotliwość występowania poszczególnych liter. Już krótkie spojrzenie na listę zestawów do gry
w Scrabble w różnych językach~\cite{wiki:scrabble} (ilość i wartość danej litery zależy od częstotliwości jej występowania
w danym języku), pozwala na stwierdzenie, że częstotliwości te są różne w różnych językach. Nie jest jednak możliwe proste
określenie reguł rozpoznawania języka na tej podstawie. Dlatego idealnym narzędziem do operowania na takich danych jest
sieć neuronowa.


\subsection{Rozpoznawanie języka na podstawie częstotliwości liter}
Niniejszy artykuł przedstawia próbę rozpoznawania języka na podstawie częstotliwości liter. Celem uproszczenia budowy
programu, zrezygnowano z analizy alfabetu i zdecydowano się na analizowanie tylko i wyłącznie częstotliwości 26~liter
alfabetu łacińskiego.

\subsubsection{Liczba wejść i wyjść}
Sieć składa się z 26 wejść: na każdym podawana jest częstotliwość występowania odpowiedniej litery alfabetu łacińskiego
w danym tekście. Liczba wyjść zależna jest od założonej ilości rozpoznawanych języków. W~przygotowanej aplikacji założono
obsługę 11 języków.

\subsubsection{Budowa sieci}
Ze względu na charakter zadania, czyli potrzebę wykrycia odpowiednich zależności w zestawie 26~sygnałów wejściowych,
zdecydowano się na wykorzystanie sieci jednokierunkowej, złożonej z trzech warstw (dodatkowa warstwa celem usprawnienia
procesu uczenia sieci). Nie istnieje żaden algorytm wspomagający podejmowanie decyzji o ilości neuronów w warstwie
ukrytej~\cite{tad:elem_wpr}. Decyzja ta jest zawsze podejmowana arbitralnie i następnie weryfikowana w czasie testów.
Należy tylko pamiętać, aby nie była ani zbyt duża, ani zbyt mała. Zdecydowano się przyjąć, że będzie się ona składała
z 10 neuronów.

\subsubsection{Funkcja aktywacji}
Każdy neuron wykorzystuje sigmoidalną funkcję aktywacji. Pozwala ona na lepsze różnicowanie zbliżonych sygnałów niż funkcja
liniowa. W rozważanym problemie, pozwala na lepsze różnicowanie zbliżonych częstotliwości występowania liter.

\subsubsection{Uczenie sieci}
Do uczenia sieci wykorzystano algorytm Resilient Propagation. Jest to zmodyfikowana wersja podstawowego algorytmu uczenia
z nauczycielem sieci wielowarstwowych jednokierunkowych, czyli algorytmu propagacji wstecznej. Zdecydowano się na ten
algorytm, ponieważ jest on najbardziej efektywnym algorytmem uczenia dla prostych sieci neuronowych, takich jak nasza~\cite{encog:rprop}.



% An example of a floating figure using the graphicx package.
% Note that \label must occur AFTER (or within) \caption.
% For figures, \caption should occur after the \includegraphics.
% Note that IEEEtran v1.7 and later has special internal code that
% is designed to preserve the operation of \label within \caption
% even when the captionsoff option is in effect. However, because
% of issues like this, it may be the safest practice to put all your
% \label just after \caption rather than within \caption{}.
%
% Reminder: the "draftcls" or "draftclsnofoot", not "draft", class
% option should be used if it is desired that the figures are to be
% displayed while in draft mode.
%
%\begin{figure}[!t]
%\centering
%\includegraphics[width=2.5in]{myfigure}
% where an .eps filename suffix will be assumed under latex, 
% and a .pdf suffix will be assumed for pdflatex; or what has been declared
% via \DeclareGraphicsExtensions.
%\caption{Simulation Results}
%\label{fig_sim}
%\end{figure}

% Note that IEEE typically puts floats only at the top, even when this
% results in a large percentage of a column being occupied by floats.


% An example of a double column floating figure using two subfigures.
% (The subfig.sty package must be loaded for this to work.)
% The subfigure \label commands are set within each subfloat command, the
% \label for the overall figure must come after \caption.
% \hfil must be used as a separator to get equal spacing.
% The subfigure.sty package works much the same way, except \subfigure is
% used instead of \subfloat.
%
%\begin{figure*}[!t]
%\centerline{\subfloat[Case I]\includegraphics[width=2.5in]{subfigcase1}%
%\label{fig_first_case}}
%\hfil
%\subfloat[Case II]{\includegraphics[width=2.5in]{subfigcase2}%
%\label{fig_second_case}}}
%\caption{Simulation results}
%\label{fig_sim}
%\end{figure*}
%
% Note that often IEEE papers with subfigures do not employ subfigure
% captions (using the optional argument to \subfloat), but instead will
% reference/describe all of them (a), (b), etc., within the main caption.


% An example of a floating table. Note that, for IEEE style tables, the 
% \caption command should come BEFORE the table. Table text will default to
% \footnotesize as IEEE normally uses this smaller font for tables.
% The \label must come after \caption as always.
%
%\begin{table}[!t]
%% increase table row spacing, adjust to taste
%\renewcommand{\arraystretch}{1.3}
% if using array.sty, it might be a good idea to tweak the value of
% \extrarowheight as needed to properly center the text within the cells
%\caption{An Example of a Table}
%\label{table_example}
%\centering
%% Some packages, such as MDW tools, offer better commands for making tables
%% than the plain LaTeX2e tabular which is used here.
%\begin{tabular}{|c||c|}
%\hline
%One & Two\\
%\hline
%Three & Four\\
%\hline
%\end{tabular}
%\end{table}


% Note that IEEE does not put floats in the very first column - or typically
% anywhere on the first page for that matter. Also, in-text middle ("here")
% positioning is not used. Most IEEE journals use top floats exclusively.
% Note that, LaTeX2e, unlike IEEE journals, places footnotes above bottom
% floats. This can be corrected via the \fnbelowfloat command of the
% stfloats package.


\section{Implementacja}
\IEEEPARstart{S}{tworzona} aplikacja składa się z trzech głównych części: obsługi plików, obsługi sieci neuronowej i GUI.



\section{Testy}
\IEEEPARstart{P}{rzeprowadzone} testy. Wyniki. Informacje o częstotliwości liter, ale też o skuteczności sieci. Wykresy? \\
Co nam wyszło, dla których języków działa, które okazały się podobne, etc

co nie wyszlo w pracy (ma byc krytyczne) problemy z implementacja, z siecia, z dokumentacja itd itp

\section{Wnioski}
\IEEEPARstart{P}{rojekt} jest fajny, bądź nie i jak go można dalej rozwijać, albo gdzie go można zastosować





% if have a single appendix:
%\appendix[Proof of the Zonklar Equations]
% or
%\appendix  % for no appendix heading
% do not use \section anymore after \appendix, only \section*
% is possibly needed

% use appendices with more than one appendix
% then use \section to start each appendix
% you must declare a \section before using any
% \subsection or using \label (\appendices by itself
% starts a section numbered zero.)
%


% Can use something like this to put references on a page
% by themselves when using endfloat and the captionsoff option.
\ifCLASSOPTIONcaptionsoff
  \newpage
\fi


% trigger a \newpage just before the given reference
% number - used to balance the columns on the last page
% adjust value as needed - may need to be readjusted if
% the document is modified later
%\IEEEtriggeratref{8}
% The "triggered" command can be changed if desired:
%\IEEEtriggercmd{\enlargethispage{-5in}}

% references section

% can use a bibliography generated by BibTeX as a .bbl file
% BibTeX documentation can be easily obtained at:
% http://www.ctan.org/tex-archive/biblio/bibtex/contrib/doc/
% The IEEEtran BibTeX style support page is at:
% http://www.michaelshell.org/tex/ieeetran/bibtex/
%\bibliographystyle{IEEEtran}
% argument is your BibTeX string definitions and bibliography database(s)
%\bibliography{IEEEabrv,../bib/paper}
%
% <OR> manually copy in the resultant .bbl file
% set second argument of \begin to the number of references
% (used to reserve space for the reference number labels box)
\begin{thebibliography}{1}

\bibitem{smykowski:jak_google_rozpoznaje}
T.~Smykowski, \emph{Jak Google rozpoznaje język tekstu?},
http://polishwords.com.pl/blog/2009/rozpoznawanie-jezyka-tekstu/,
(dostęp: 2012-05-31)

\bibitem{tad:elem_wpr}
R.~Tadeusiewicz, \emph{Elementarne wprowadzenie do techniki sieci neuronowych
z przykładowymi programami}, Warszawa, Polska: Akademicka Oficyna Wydawnicza PLJ, 1998.

\bibitem{tad:sieci}
R.~Tadeusiewicz, \emph{Sieci neuronowe}, wyd.~2, Warszawa, Polska: Akademicka
Oficyna Wydawnicza RM, 1993.

\bibitem{ethnologue}
\emph{Ethnologue}, http://www.ethnologue.com/language\_index.asp,
(dostęp: 2012-05-31)

\bibitem{wiki:langs}
\emph{Language recognition chart}, \\ http://en.wikipedia.org/wiki/Wikipedia:Language\_recognition\_chart,
(dostęp: 2012-05-31)

\bibitem{encog:rprop}
\emph{Resilient Propagation}, \\ http://www.heatonresearch.com/wiki/Resilient\_Propagation,
(dostęp: 2012-05-31)

\bibitem{wiki:scrabble}
\emph{Scrabble letter distributions}, \\ http://en.wikipedia.org/wiki/Scrabble\_letter\_distributions,
(dostęp: 2012-05-31)

\bibitem{wiki:sieci}
\emph{Sieć neuronowa}, http://pl.wikipedia.org/wiki/Sieć\_neuronowa,
(dostęp: 2012-05-31)


\end{thebibliography}

\end{document}

