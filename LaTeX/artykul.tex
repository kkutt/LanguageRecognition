\documentclass[journal]{IEEEtran}

%\usepackage{polski}
\usepackage[T1]{fontenc}
\usepackage[utf8]{inputenc}
\usepackage[pdftex]{graphicx}
\graphicspath{/img/}

% correct bad hyphenation here
\hyphenation{op-tical net-works semi-conduc-tor}


\begin{document}
%
% paper title
% can use linebreaks \\ within to get better formatting as desired
\title{Rozpoznawanie języka tekstu \\ na podstawie częstotliwości liter}

\author{Krzysztof~Kutt, Michał~Nowak
\thanks{K. Kutt i M. Nowak, Katedra Automatyki, Wydział Elektrotechniki, Automatyki, Informatyki i Elektroniki,
Akademia Górniczo-Hutnicza, Kraków, Polska, e-mail: kkutt@student.agh.edu.pl, mnowak@student.agh.edu.pl}}

% make the title area
\maketitle


\begin{abstract}
TODO: Tutaj leci abstrakt
\end{abstract}

% Note that keywords are not normally used for peerreview papers.
\begin{IEEEkeywords}
sieć neuronowa, język, rozpoznawanie.
\end{IEEEkeywords}


\section{Wprowadzenie}
% The very first letter is a 2 line initial drop letter followed
% by the rest of the first word in caps.
% 
% form to use if the first word consists of a single letter:
% \IEEEPARstart{A}{demo} file is ....
% 
% form to use if you need the single drop letter followed by
% normal text (unknown if ever used by IEEE):
% \IEEEPARstart{A}{}demo file is ....
% 
% Some journals put the first two words in caps:
% \IEEEPARstart{T}{his demo} file is ....
% 
% Here we have the typical use of a "T" for an initial drop letter
% and "HIS" in caps to complete the first word.
\IEEEPARstart{P}{rzedstawiona} w tym artykule aplikacja powstała w~ramach przedmiotu
Sztuczne Sieci Neuronowe, semestr letni 2011/2012, prowadzący: mgr inż. Tomasz Orzechowski.
% You must have at least 2 lines in the paragraph with the drop letter
% (should never be an issue)

\subsection{Sieci neuronowe}
Sztuczne Sieci Neuronowe to modele matematyczne, stworzone na wzór naturalnych połączeń neuronów
znajdujących się w ludzkim mózgu. Każdy sztuczny neuron zbiera sygnały wejścia, dokonuje na nich
odpowiedniej transformacji i zwraca odpowiednią wartość na wyjściu. Sztuczna sieć neuronowa może
składać się z tysięcy takich elementarnych neuronów. W zależności od budowy sieci oraz od wykorzystywanej
przez neurony funkcji aktywacji, wyróżniamy różne rodzaje sieci.

% needed in second column of first page if using \IEEEpubid
%\IEEEpubidadjcol

\subsection{Rodzaje sieci}
Skorzystać z książki Tadeusiewicza~\cite{tad:sieci} - główne rodzaje sieci to rozdziały książki - napisać
zdanie-dwa o każdym z nich


\subsection{Zastosowania sieci}
Wiki i książka tadeusiewicza.


\subsection{Rozpoznawanie języka}
Coś ogólnie o naszym problemie. Do czego to może być potrzebne, kto już walczył z tym problemem i jak
(np google translator ma opcję rozpoznawania języka. \\
Przedstawione opcje to nie musi być tylko częstotliwość liter - pewnie są jakieś inne (choćby sprawdzanie,
czy występują znaki specjalne danego języka :P)


\subsection{Określenie języka przy pomocy sieci neuronowej}
Czyli wprowadzenie teoretyczne do naszego projektu. Uznaliśmy, że ten problem można rozwiązać przy pomocy sieci
neuronowej. Uznaliśmy, że fajnie będzie to zrobić przy pomocy prostej sieci, złożonej z trzech wartstw,
z neuronami z aktywacją sigmoidalną (i dlaczego akurat taką :P)\\
Do środkowej warstwy wzięliśmy 10 neuronów - tak z dupy trochę - może to potestować? (i wtedy w testach
informcję o tym, że zweryfikowaliśmy wstępne założenia i okazały się dobre / złe - musieliśmy zmienić
drastycznie liczbę neuronów, etc.)\\
W książce~\cite{tad:elem_wpr} są informacje o tym jak wybierać sieć (na tej podstawie ustalić np wstępną liczbę
neuronów)


% An example of a floating figure using the graphicx package.
% Note that \label must occur AFTER (or within) \caption.
% For figures, \caption should occur after the \includegraphics.
% Note that IEEEtran v1.7 and later has special internal code that
% is designed to preserve the operation of \label within \caption
% even when the captionsoff option is in effect. However, because
% of issues like this, it may be the safest practice to put all your
% \label just after \caption rather than within \caption{}.
%
% Reminder: the "draftcls" or "draftclsnofoot", not "draft", class
% option should be used if it is desired that the figures are to be
% displayed while in draft mode.
%
%\begin{figure}[!t]
%\centering
%\includegraphics[width=2.5in]{myfigure}
% where an .eps filename suffix will be assumed under latex, 
% and a .pdf suffix will be assumed for pdflatex; or what has been declared
% via \DeclareGraphicsExtensions.
%\caption{Simulation Results}
%\label{fig_sim}
%\end{figure}

% Note that IEEE typically puts floats only at the top, even when this
% results in a large percentage of a column being occupied by floats.


% An example of a double column floating figure using two subfigures.
% (The subfig.sty package must be loaded for this to work.)
% The subfigure \label commands are set within each subfloat command, the
% \label for the overall figure must come after \caption.
% \hfil must be used as a separator to get equal spacing.
% The subfigure.sty package works much the same way, except \subfigure is
% used instead of \subfloat.
%
%\begin{figure*}[!t]
%\centerline{\subfloat[Case I]\includegraphics[width=2.5in]{subfigcase1}%
%\label{fig_first_case}}
%\hfil
%\subfloat[Case II]{\includegraphics[width=2.5in]{subfigcase2}%
%\label{fig_second_case}}}
%\caption{Simulation results}
%\label{fig_sim}
%\end{figure*}
%
% Note that often IEEE papers with subfigures do not employ subfigure
% captions (using the optional argument to \subfloat), but instead will
% reference/describe all of them (a), (b), etc., within the main caption.


% An example of a floating table. Note that, for IEEE style tables, the 
% \caption command should come BEFORE the table. Table text will default to
% \footnotesize as IEEE normally uses this smaller font for tables.
% The \label must come after \caption as always.
%
%\begin{table}[!t]
%% increase table row spacing, adjust to taste
%\renewcommand{\arraystretch}{1.3}
% if using array.sty, it might be a good idea to tweak the value of
% \extrarowheight as needed to properly center the text within the cells
%\caption{An Example of a Table}
%\label{table_example}
%\centering
%% Some packages, such as MDW tools, offer better commands for making tables
%% than the plain LaTeX2e tabular which is used here.
%\begin{tabular}{|c||c|}
%\hline
%One & Two\\
%\hline
%Three & Four\\
%\hline
%\end{tabular}
%\end{table}


% Note that IEEE does not put floats in the very first column - or typically
% anywhere on the first page for that matter. Also, in-text middle ("here")
% positioning is not used. Most IEEE journals use top floats exclusively.
% Note that, LaTeX2e, unlike IEEE journals, places footnotes above bottom
% floats. This can be corrected via the \fnbelowfloat command of the
% stfloats package.


\section{Implementacja}
Tutaj idzie sekcja o tym jak podzieliliśmy na moduły, jaką sieć wybraliśmy, jaką bibliotekę (i dlaczego)



\section{Testy}
Przeprowadzone testy. Wyniki. Informacje o częstotliwości liter, ale też o skuteczności sieci. Wykresy? \\
Co nam wyszło, dla których języków działa, które okazały się podobne, etc



\section{Wnioski}
Projekt jest fajny, bądź nie i jak go można dalej rozwijać, albo gdzie go można zastosować





% if have a single appendix:
%\appendix[Proof of the Zonklar Equations]
% or
%\appendix  % for no appendix heading
% do not use \section anymore after \appendix, only \section*
% is possibly needed

% use appendices with more than one appendix
% then use \section to start each appendix
% you must declare a \section before using any
% \subsection or using \label (\appendices by itself
% starts a section numbered zero.)
%


% Can use something like this to put references on a page
% by themselves when using endfloat and the captionsoff option.
\ifCLASSOPTIONcaptionsoff
  \newpage
\fi


% trigger a \newpage just before the given reference
% number - used to balance the columns on the last page
% adjust value as needed - may need to be readjusted if
% the document is modified later
%\IEEEtriggeratref{8}
% The "triggered" command can be changed if desired:
%\IEEEtriggercmd{\enlargethispage{-5in}}

% references section

% can use a bibliography generated by BibTeX as a .bbl file
% BibTeX documentation can be easily obtained at:
% http://www.ctan.org/tex-archive/biblio/bibtex/contrib/doc/
% The IEEEtran BibTeX style support page is at:
% http://www.michaelshell.org/tex/ieeetran/bibtex/
%\bibliographystyle{IEEEtran}
% argument is your BibTeX string definitions and bibliography database(s)
%\bibliography{IEEEabrv,../bib/paper}
%
% <OR> manually copy in the resultant .bbl file
% set second argument of \begin to the number of references
% (used to reserve space for the reference number labels box)
\begin{thebibliography}{1}

\bibitem{tad:sieci}
R.~Tadeusiewicz, \emph{Sieci neuronowe}, wyd.~2, Warszawa, Polska: Akademicka
Oficyna Wydawnicza RM, 1993.

\bibitem{tad:elem_wpr}
R.~Tadeusiewicz, \emph{Elementarne wprowadzenie do techniki sieci neuronowych
z przykładowymi programami}, Warszawa, Polska: Akademicka Oficyna Wydawnicza PLJ, 1998.


\end{thebibliography}

\end{document}

